\documentclass[a4paper, 10pt]{article}

% Packages:
\usepackage[
    ignoreheadfoot, % set margins without considering header and footer
    top=1.0 cm, % seperation between body and page edge from the top
    bottom=1.0 cm, % seperation between body and page edge from the bottom
    left=1 cm, % seperation between body and page edge from the left
    right=1 cm, % seperation between body and page edge from the right
    % showframe % for debugging 
]{geometry} % for adjusting page geometry
\usepackage[explicit]{titlesec} % for customizing section titles
\usepackage[dvipsnames]{xcolor} % for coloring text
\definecolor{primaryColor}{RGB}{0, 79, 144} % define primary color
\usepackage{enumitem} % for customizing lists
\usepackage{fontawesome5} % for using icons
\usepackage{amsmath} % for math
\usepackage[
    pdftitle={Roméo Maignal's CV},
    pdfauthor={Roméo Maignal},
    pdfcreator={LaTeX with RenderCV},
    colorlinks=true,
    urlcolor=primaryColor
]{hyperref} % for links, metadata and bookmarks
\usepackage[pscoord]{eso-pic} % for floating text on the page
\usepackage{calc} % for calculating lengths
\usepackage{bookmark} % for bookmarks
\usepackage{lastpage} % for getting the total number of pages
\usepackage{changepage} % for one column entries (adjustwidth environment)
\usepackage{paracol} % for two and three column entries
\usepackage{ifthen} % for conditional statements
\usepackage{needspace} % for avoiding page brake right after the section title
\usepackage{iftex} % check if engine is pdflatex, xetex or luatex
\usepackage{multicol} % for multiple columns

% Ensure that generate pdf is machine readable/ATS parsable:
\ifPDFTeX
    \input{glyphtounicode}
    \pdfgentounicode=1
    \usepackage[T1]{fontenc}
    \usepackage[utf8]{inputenc}
    \usepackage{lmodern}
\fi

\usepackage[default, type1]{sourcesanspro} 

% Some settings:
\AtBeginEnvironment{adjustwidth}{\partopsep0pt} % remove space before adjustwidth environment
\pagestyle{empty} % no header or footer
\setcounter{secnumdepth}{0} % no section numbering
\setlength{\parindent}{0pt} % no indentation
\setlength{\topskip}{0pt} % no top skip
\setlength{\columnsep}{0.15cm} % set column seperation
% \makeatletter
% \let\ps@customFooterStyle\ps@plain % Copy the plain style to customFooterStyle
% \patchcmd{\ps@customFooterStyle}{\thepage}{
%     \color{gray}\textit{\small Roméo Maignal - Page \thepage{} of \pageref*{LastPage}}
% }{}{} % replace number by desired string
% \makeatother
% \pagestyle{customFooterStyle}
\titleformat{\section}{
    % avoid page braking right after the section title
    \needspace{4\baselineskip}
    % make the font size of the section title large and color it with the primary color
    \Large\color{primaryColor}
}{
}{
}{
    % print bold title, give 0.15 cm space and draw a line of 0.8 pt thickness
    % from the end of the title to the end of the body
    \textbf{#1}\hspace{0.15cm}\titlerule[0.8pt]\hspace{-0.1cm}
}[] % section title formatting

\titlespacing{\section}{
    % left space:
    -1pt
}{
    % top space:
    0.2 cm
}{
    % bottom space:
    0.1 cm
} % section title spacing

% \renewcommand\labelitemi{$\vcenter{\hbox{\small$\bullet$}}$} % custom bullet points
\newenvironment{highlights}{
    \begin{itemize}[
        topsep=0.10 cm,
        parsep=0.10 cm,
        partopsep=0pt,
        itemsep=0pt,
        leftmargin=0.4 cm + 10pt
    ]
}{
    \end{itemize}
} % new environment for highlights




\newenvironment{header}{
    \setlength{\topsep}{0pt}\par\kern\topsep\centering\color{primaryColor}\linespread{1.5}
}{
    \par\kern\topsep
} % new environment for the header

\newcommand{\placelastupdatedtext}{% \placetextbox{<horizontal pos>}{<vertical pos>}{<stuff>}
  \AddToShipoutPictureFG*{% Add <stuff> to current page foreground
    \put(
        \LenToUnit{\paperwidth-2 cm-0.2 cm+0.05cm},
        \LenToUnit{\paperheight-1.0 cm}
    ){}%
  }%
}%

% save the original href command in a new command:
\let\hrefWithoutArrow\href

% new command for external links:
\renewcommand{\href}[2]{\hrefWithoutArrow{#1}{\ifthenelse{\equal{#2}{}}{ }{#2 }\raisebox{.15ex}{\footnotesize \faExternalLink*}}}


\begin{document}
    \newcommand{\AND}{\unskip
        \cleaders\copy\ANDbox\hskip\wd\ANDbox
        \ignorespaces
    }
    \newsavebox\ANDbox
    \sbox\ANDbox{}

    \placelastupdatedtext
    \begin{header}
        \fontsize{20 pt}{20 pt}
        \textbf{Roméo Maignal}
        
        \vspace{0.2 cm}

        \normalsize
        \mbox{\hrefWithoutArrow{https://github.com/relogamimano}{{\footnotesize\faGithub}\hspace*{0.13cm}relogamimano}}%
        \kern 0.25 cm%
        \AND%
        \kern 0.25 cm%
        \mbox{\hrefWithoutArrow{https://www.linkedin.com/in/rom\%C3\%A9o-maignal-5a62a21a5/}{{\footnotesize\faLinkedinIn}\hspace*{0.13cm}roméo-maignal}}%
        \kern 0.25 cm%
        \AND%
        \kern 0.25 cm%
        \mbox{\hrefWithoutArrow{mailto:romeo.maignal@epfl.ch}{{\footnotesize\faEnvelope[regular]}\hspace*{0.13cm}romeo.maignal@epfl.ch}}%
        \kern 0.25 cm%
        \AND%
        \kern 0.25 cm%
        \mbox{\hrefWithoutArrow{tel:+33-6-95-75-99-98}{{\footnotesize\faPhone*}\hspace*{0.13cm}06 95 75 99 98}}%
        \kern 0.25 cm%
        \AND%
        \kern 0.25 cm%
        \mbox{\hrefWithoutArrow{https://www.google.com/maps/place/Lausanne/}{{\footnotesize\faMapMarker*}\hspace*{0.13cm}Lausanne, Switzerland}}%
    \end{header}
    
    \vspace{0.5 cm}
    I am a bachelor student in Computer Science at EPFL in Switzerland. Previously, I've worked as a system engineer for a CubeSat mission within the EPFL Spacecraft Team. 
    My interests include open-source softwares, systems engineering, low-level programming and cyber-security. 
    % I have many interests but when it comes to coding, I spend most of my time collaborating on open-source projects and making sure all of my work always has a nice UI.
    
    \section{Work Experience}
        \textbf{EPFL Spacecraft Team}\href{https://www.epflspacecraftteam.ch/}{}\hfill Lausanne, Switzerland\\
        {System Engineer for the CHESS Mission} \hfill Sept 2023 - Febr 2025
        \begin{highlights}
            \item Contributed to the mission's PDR by writing the Interface Control Document for 4 subsystems following ECSS standards.
            \item Defined system requirements for CubeSat data/power transmission and the payload’s architecture, weight and dimensions.
            \item Conducted tests and simulations to verify the performance and reliability of the telecommunication components.
            \item Successfully troubleshot and resolved technical issues that arose during the satellite's design review such as misplacement of crucial mechanical interfaces.
        \end{highlights}
        
    \section{Education}
        \textbf{École Polytechnique Fédérale de Lausanne}\hfill Lausanne, Switzerland\\
        Bachelor in Computer Science\hfill Sept 2022 - June 2026
        \begin{highlights}
            \item Relevant courses : Computer Architecture, Computer Systems \& Network, Intro to ML, Software Construction, Digital System Design, Object Oriented Programming, 
            Data-Intensive Systems, Computer Language Processing, Human-Computer Interaction 
            
        \end{highlights}
        \textbf{Lycée Français International Georges Pompidou} \hfill Dubai, UAE\\
        Baccalauréat Général\hfill Sept 2016 - June 2022
        \begin{highlights}
            \item Relevant courses: Mathématiques, Physique-Chimie, Mathes Expertes
            
        \end{highlights}
    
    \section{Notable Projects}
    % \vspace{-2em} % Adjust the value as needed
    % \begin{multicols}{2}    

        \textbf{Aircraft Flights Tracking Software}\href{https://github.com/relogamimano/Javions}{}
        \begin{highlights}
            \item Developed a FlightRadar24-like software from scratch in Java over a period of four months. It involved various
            engineering fields (data transmission, aircraft data interfaces, ADS-B communication, aircraft location settings, cryptography).
            \item Tools used: Java, JavaFX, JUnit, JVM, IntelliJ, ADS-B Antenna
        \end{highlights}

        \textbf{Conway's Game-of-Life in RISC-V Assembly}\href{https://github.com/relogamimano/Conway-s-Game-of-Life-in-RISC-V-Assembly}{}
        \begin{highlights}
            \item Implementation, on a GECKO board, of John Conway's famous Game of Life, in RISC-V assembly language.
            \item Tools used: RISC-V, GTKWave, MemoryView
        \end{highlights}
        \textbf{Scala Web App}\href{https://github.com/relogamimano/Scala-Web-App}{}
        \begin{highlights}
            \item Multiplayer web game application written in Scala. It Implements a famous board game and use a client-server architecture to connect multiple players. I made use of scala state machines and json serialization/deserialization functions.
            \item Tools used: Scala, Metals, SBT
        \end{highlights}

        \textbf{Interface Control Documents}\href{https://drive.google.com/drive/folders/1KWNjMkGwcYpTIh6xBs-nddBK195R_ihb?usp=sharing}{}
        \begin{highlights}
            \item ICDs for the telecommunication subsystems of a CubeSat. It involved learning the specific requirements for CubeSat development and adhering to ECSS standards to give the project good technical directives.
            \item Tools used: Overleaf, LaTeX, FusionCAD
        \end{highlights}

        \textbf{Haystack Store File System}\href{https://github.com/relogamimano/haystack-store}{}
        \begin{highlights}
            \item Scalable file system for big data, inspired by Facebook's Haystack Storage system. It offers a cheap and higher performance solution to the user image storage space problem.
            \item Tools used: C, GCC
        \end{highlights}


        % \textbf{Roguelike Java Game}\href{https://github.com/relogamimano/Rogue-Like-Game-MP2-2022}{}
        % \begin{highlights}
        %     \item Roguelike desktop game built in Java.
        %     \item Tools used: Java, JavaFX, JUnit, IntelliJ
        % \end{highlights}


        % \textbf{Image Compressor / Decompressor}\href{https://github.com/relogamimano/Image-Compressor-Decompressor}{}
        % \begin{highlights}
        %     \item Java program using QOI format to compress and decompress images.
        %     \item Tools used: Java, JUnit, IntelliJ
        % \end{highlights}
    % \end{multicols}
    % \vspace{-1em} % Adjust the value as needed

        
        
    
    \section{Competences}
    \textbf{Languages:} C, Java, C++, Scala, Python, Javascript/Typescript, RISC-V Assembly, Verilog, VHDL, LaTeX
    
    \vspace{0.2 cm}
    
    \textbf{Tools:} Git, Github, VSCode, IntelliJ, Overleaf, GTKWave, OnShape, Blender ; \textbf{Frameworks/Platforms:} React, Next.js, Node.js
    
    \vspace{0.2cm}
    
    \textbf{Skills:} Functional Programming, Object-Oriented Programming, Memory and Network Oriented Programming, Algorithms, Data Structures, Machine Learning
    
    
    
    \section{Non-Technical Work Experience}
    % % \textbf{TownSquare}\hfill Dubai\\
    % % Piano Teacher \hfill Sept 2020 - April 2021
    % % \begin{highlights}
    % %     \item Taught piano and music theory to young children, fostering their musical development.
    % % \end{highlights}
    
    \textbf{CSNB}\hfill Brive, France\\
    Seasonal Boat Rental Manager\hfill July 2020 - July 2023
    \begin{highlights}
        \item Summer job at a lake sports facility. Rented and repaired various boats. Led summer
        camp children in different activities.
    \end{highlights}
    
    % \textbf{Rolex Learning Center}\hfill Lausanne, Switzerland\\
    % Klee barista\hfill June 2023 - June 2023
    \end{document}